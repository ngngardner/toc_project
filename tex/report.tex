\documentclass[12pt]{article}
\usepackage[T1]{fontenc}
\usepackage[utf8]{inputenc}
\usepackage{authblk}

\title{Improved Macroscopic Traffic Flow Modelling with Cellular Automata}
\author[1]{Andy Vu}
\author[1]{Oliver Tran}
\author[1]{Sanjeev Khemani}
\author[1]{Noah Gardner}
\affil[1]{College of Computing and Software Engineering, Kennesaw State University}
\date{\today}
\renewcommand\Authands{ and }

\begin{document}
\maketitle
\section{Abstract}
This paper is based on another research paper in cellular automata in which city
traffic and its affection by factors like distributed occurrence, evanescence,
and time variation is observed on a two-dimensional grid. The variables used in
this new experiment include diagonal paths, one-way streets, pre-generated
flows, varying speeds, varying destination priority, and different types of
roads. With these additional factors in the experiment, the results of the new
simulation will yield results that more accurately reflect the environment of
real-world traffic.

\section{Research Statement}
There are a few interesting changes we could make to the original algorithm in
order to explore the impact of different traffic environments and flows. For
example, the original paper randomly generates \textit{traffic flows} which are
an abstracted representation of the traffic that really represents a group of
vehicles as a random amount of traffic. Additionally, when each traffic flow is
generated, the destination of the flow is also generated randomly. After the
traffic flow is generated and the destination is chosen, the traffic flow
chooses it's path automatically as the shortest distance from the starting cell
to the current cell. Finally, the original paper uses one traffic environment
with simple two-way roads that are placed on the grid laterally and vertically.

One idea we would like to explore is to add different priorities for cells as
destinations. If a cell is a high priority destination, then it is more likely
to be chosen as a destination for a traffic flow when it is generated. Another
improvement could also allow for modelling of different types of roads. For
example, rather than the flow taking the shortest path from the source to the
destination, each cell could have a parameter for the weight of adding that cell
to the path of the traffic flow. If the weight of adding the cell to the path is
low, that cell may be modeled as part of a road that can handle more traffic
such as a highway. The original experiment also did not test diagonal or one-way
streets, so that may be valuable for a traffic flow model as well.

\section{Literature Review}
\begin{itemize}
    \item Simulation of urban macro-traffic flow based on cellular automata
          \cite{pang_simulation_2019}
          \begin{description}
              \item This paper uses a cellular automata model to simulate
                    traffic flow on a two-dimensional grid. Traffic flows are
                    randomly generated that have a certain amount of traffic
                    (low or heavy) and also have a random destination. There
                    exists a grid with cells filled in which are meant to model
                    the roads for the traffic flow to move on. On each timestep,
                    the traffic flows move towards their destination using a
                    minimum distance method. The cells can hold a certain amount
                    of traffic capacity. When the traffic capacity for a cell is
                    reached, the cell will not allow new traffic flows to enter
                    the cell. The cell also have a traffic volume, which helps
                    the programmer to see at a glance which cells have high and
                    low volumes of traffic.
          \end{description}
    \item A cellular automaton model for freeway traffic
          \cite{nagel_cellular_1992}
          \begin{description}
              \item Nagel and Schreckenberg have gone away from the typical
                    fluid dynamical approaches to traffic flow studies during
                    this time to the use of a more revoluationary technique. The
                    need for this was to introduce a factor of laminar
                    start-stop waves with increasing vehicle density with an
                    emphasis on human behavior a traffic environment. In their
                    model, it is represented as a single dimensional array where
                    each cell may be a vehicle with functions to accelearte,
                    slow down, as well as randomized velocities. These functions
                    are to represent the randomness in human behavior and
                    varying conditions that are expected during traffic. The
                    authors concluded that a computational way to simulate
                    traffic flow, was more advantagous to a fluid dynamical
                    approach that incorporated real world behavior ofa driver
                    but also retained the key aspects of fluid dynamic studies.
          \end{description}
    \item Self-organization and phase transition in traffic-flow model of a
          two-lane roadway \cite{nagatani_self-organization_1993}
          \begin{description}
              \item Nagatani implements a two lane model as an extension of a one dimensional cellular automaton to monitor lane changing. Their model is essentially two 1-dimensional lattaces
              to represent a two lane roadway. Rules using arrows within the lattace were deteremined to allow cars to change lanes at certain times such as when a car infront was blocking
              the path to progress forward from either the right to left direcetion or vice versa. These rules were dependent on the time step. Overall the results showed that as the cars
              general velocity increased, there was a usage of lane shifting without cars obstructing progress. As density increases, the maximal velocity decreases. There was a "sweet-spot"
              indicating the optimal amount of lane changing within a set of density of traffic where lane changing peaked with a critical value of density. If the desnity fell or rose
              too high, this optimal amount of lane changing dropped. 
          \end{description}
    \item Multi-value cellular automata model for mixed bicycle flow
          \cite{jia_multi-value_2007}
          \begin{description}
              \item Instead of car and vehicle traffic like the other papers, this paper focuses on mixed bicycle flow using a multi valued cellular automata. Here the authors implemented two bicycles
              with different maximum speeds. The usage of bicycles to compare to vehicles have an emphasis on the behavioral and personality aspect that can be difficult to analyze with cars.
              The authors explain that young riders tend to ride at high speeds while older riders ride at a lower speed. These differences show that there is no maxium velocity. In this 
              model, bicycles can move to their next open site with faster bicycles moving with priority over slower bicycles. It was shown similarily to vehicles that slow bicycles that 
              congregate and occupy sites, block faster moving bicycles that cannot overtake and thus move in a platoon order. Increasing slow bicycle density will congest and bottleneck
              the simulation. One of the points that B. Jia and the authors mention that cannot be properly analysed is the nature of riders that group together because they are classmates
              or friends thus this simulation does have some flaws. Increasing randomization of the locations and density of slow riders tended to allow more free flow and less bottlenecking.
          \end{description}
    \item Cellular automaton model for bidirectional traffic
          \cite{simon_cellular_1998-1}
          \begin{description}
              \item Simon and Gutowitz have built upon the paper referenced
                    above by Nagel and Schreckenberg. Here they are implementing
                    a bidrectional traffic using a two lane road with traffic
                    moving in opposite direction in comparison to Nagel and
                    Schreckenberg's one lane cellular automaton. They have
                    incorporated inteteractions to simulate passing, as well as
                    a distribution of varying vehicle speeds. Once again, Simon
                    and gutowitz, like their reference model, had an emphasis on
                    approximating the behavior of real traffic and human
                    behavior. They wanted to address the issue of a one lane
                    cellular atomaton where all vehicles have a maximum velocity
                    and thus the model unrealistically follows a lead slow car,
                    hence the need and emphasis on passing. The bidirectional
                    model varried in types where there was varrying rules
                    regarding passing. The researchers have found that passing
                    greatly increases fluidity in traffic not seen in single
                    dimensional cellular automaton and greatly resembles real
                    world traffic dynamics and flow.
          \end{description}
\end{itemize}

\section{Methods}
For this project, we plan to use Python to implement our cellular automata
traffic flow model. We may also include a GUI to visualize the traffic flow on
each time step. This section will be expanded as the project progresses.

\section{Milestones}

\begin{tabular}{|l|l|}
    \hline
    \textbf{Date} & \textbf{Milestone}                                             \\
    \hline
    $9/24/2021$   & \textbf{Project Deliverable 1}                                 \\
    \hline
                  & Implement the original cellular automata and algorithm         \\
    \hline
                  & Add weights to the cells to model different types of roads     \\
    \hline
                  & Add destination priorities to model different types of traffic \\
    \hline
                  & Model different traffic environments (one-way, etc.)
    \\
    \hline
    $10/29/2021$  & \textbf{Project Deliverable 2}
    \\
    \hline
                  & Create different environments to run our proposed algorithm
    \\
    \hline
                  & Collect experimental results and present findings
    \\
    \hline
                  & \textit{Optional} Add GUI to visualize the traffic flow
    \\
    \hline
\end{tabular}


\section{Project Management}
Our project has already set up weekly meetings in order to share progress and
updates, and we schedule ad-hoc meetings when necessary. Additionally, we have
our project inside of a github repository, where we can easily share our updated
code. We plan to make use of pair-programming sessions, where we can collaborate
while also sharing the load of programming.

\begin{itemize}
    \item \textbf{Members' Roles}
    \item[-] \textbf{Andy Vu:} Programming, Writing
    \item[-] \textbf{Olivier Tran:} Programming, Writing, Submitting Documents
    \item[-] \textbf{Sanjeev Khemani:} Programming, Writing
    \item[-] \textbf{Noah Gardner:} Programming, Writing, Source Control
\end{itemize}

\bibliography{report}
\bibliographystyle{unsrt}

\end{document}