\documentclass[sigplan,screen]{acmart}

%%
%% \BibTeX command to typeset BibTeX logo in the docs
\AtBeginDocument{%
  \providecommand\BibTeX{{%
    \normalfont B\kern-0.5em{\scshape i\kern-0.25em b}\kern-0.8em\TeX}}}

%% Rights management information.  This information is sent to you % when you
%complete the rights form.  These commands have SAMPLE % values in them; it is
%your responsibility as an author to replace % the commands and values with
%those provided to you when you % complete the rights form.
\setcopyright{acmcopyright}
\copyrightyear{2018}
\acmYear{2018} \acmDOI{10.1145/1122445.1122456}

%% These commands are for a PROCEEDINGS abstract or paper.
% \acmConference[Woodstock '18]{Woodstock '18: ACM Symposium on Neural Gaze
%   Detection}{June 03--05, 2018}{Woodstock, NY} \acmBooktitle{Woodstock '18: ACM
%   Symposium on Neural Gaze Detection, June 03--05, 2018, Woodstock, NY}
%   \acmPrice{15.00} \acmISBN{978-1-4503-XXXX-X/18/06}

%%
%% Submission ID. % Use this when submitting an article to a sponsored event.
%You'll % receive a unique submission ID from the organizers % of the event, and
%this ID should be used as the parameter to this command.
%%\acmSubmissionID{123-A56-BU3}

%%
%% end of the preamble, start of the body of the document source.
\begin{document}
\title{Improved Macroscopic Traffic Flow Modelling with Cellular Automata}

%%
%% The "author" command and its associated commands are used to define % the
%authors and their affiliations. % Of note is the shared affiliation of the
%first two authors, and the % "authornote" and "authornotemark" commands % used
%to denote shared contribution to the research.
\author{Noah Gardner}
\authornote{All authors contributed equally to this research.}
\email{ngardn10@students.kennesaw.edu}
\orcid{0000-0001-5900-9841}
\affiliation{%
    \institution{College of Computing and Software Engineering }
    \streetaddress{1100 South Marietta}
    \city{Marietta}
    \state{Georgia}
    \country{USA}
    \postcode{30060}
}

\author{Andy Vu}
\authornotemark[1]
\affiliation{%
    \institution{College of Computing and Software Engineering }
    \streetaddress{1100 South Marietta}
    \city{Marietta}
    \state{Georgia}
    \country{USA}
    \postcode{30060}
}

\author{Olivier Tran}
\authornotemark[1]
\affiliation{%
    \institution{College of Computing and Software Engineering }
    \streetaddress{1100 South Marietta}
    \city{Marietta}
    \state{Georgia}
    \country{USA}
    \postcode{30060}
}

\author{Sanjeev Khemani}
\authornotemark[1]
\affiliation{%
    \institution{College of Computing and Software Engineering }
    \streetaddress{1100 South Marietta}
    \city{Marietta}
    \state{Georgia}
    \country{USA}
    \postcode{30060}
}

%%
%% By default, the full list of authors will be used in the page % headers.
%Often, this list is too long, and will overlap % other information printed in
%the page headers. This command allows % the author to define a more concise
%list % of authors' names for this purpose.
\renewcommand{\shortauthors}{Trovato and Tobin, et al.}

%%
%% The abstract is a short summary of the work to be presented in the % article.
\begin{abstract}
    This paper is based on another research paper in cellular automata in which
    city traffic and its affection by factors like distributed occurrence,
    evanescence, and time variation is observed on a two-dimensional grid. The
    variables used in this new experiment include diagonal paths, one-way streets,
    pre-generated flows, varying speeds, varying destination priority, and
    different types of roads. With these additional factors in the experiment, the
    results of the new simulation will yield results that more accurately reflect
    the environment of real-world traffic.
\end{abstract}

%%
%% Keywords. The author(s) should pick words that accurately describe % the work
%being presented. Separate the keywords with commas.
\keywords{traffic flow, cellular automata}

%%
%% This command processes the author and affiliation and title % information and
%builds the first part of the formatted document.
\maketitle

\section{Research Statement}
There are a few interesting changes we could make to the original algorithm in
order to explore the impact of different traffic environments and flows. For
example, the original paper randomly generates \textit{traffic flows} which are
an abstracted representation of the traffic that really represents a group of
vehicles as a random amount of traffic. Additionally, when each traffic flow is
generated, the destination of the flow is also generated randomly. After the
traffic flow is generated and the destination is chosen, the traffic flow
chooses it's path automatically as the shortest distance from the starting cell
to the current cell. Finally, the original paper uses one traffic environment
with simple two-way roads that are placed on the grid laterally and vertically.

One idea we would like to explore is to add different priorities for cells as
destinations. If a cell is a high priority destination, then it is more likely
to be chosen as a destination for a traffic flow when it is generated. Another
improvement could also allow for modelling of different types of roads. For
example, rather than the flow taking the shortest path from the source to the
destination, each cell could have a parameter for the weight of adding that cell
to the path of the traffic flow. If the weight of adding the cell to the path is
low, that cell may be modeled as part of a road that can handle more traffic
such as a highway. The original experiment also did not test diagonal or one-way
streets, so that may be valuable for a traffic flow model as well.

\section{Literature Review}
\begin{itemize}
    \item Simulation of urban macro-traffic flow based on cellular automata
          \cite{pang_simulation_2019}
          \begin{description}
              \item This paper uses a cellular automata model to simulate traffic
                    flow on a two-dimensional grid. Traffic flows are randomly
                    generated that have a certain amount of traffic (low or heavy)
                    and also have a random destination. There exists a grid with
                    cells filled in which are meant to model the roads for the
                    traffic flow to move on. On each timestep, the traffic flows
                    move towards their destination using a minimum distance method.
                    The cells can hold a certain amount of traffic capacity. When
                    the traffic capacity for a cell is reached, the cell will not
                    allow new traffic flows to enter the cell. The cell also have a
                    traffic volume, which helps the programmer to see at a glance
                    which cells have high and low volumes of traffic.
          \end{description}
    \item A cellular automaton model for freeway traffic
          \cite{nagel_cellular_1992}
          \begin{description}
              \item Nagel and Schreckenberg have gone away from the typical fluid
                    dynamical approaches to traffic flow studies during this time to
                    the use of a more revoluationary technique. The need for this
                    was to introduce a factor of laminar start-stop waves with
                    increasing vehicle density with an emphasis on human behavior a
                    traffic environment. In their model, it is represented as a
                    single dimensional array where each cell may be a vehicle with
                    functions to accelearte, slow down, as well as randomized
                    velocities. These functions are to represent the randomness in
                    human behavior and varying conditions that are expected during
                    traffic. The authors concluded that a computational way to
                    simulate traffic flow, was more advantagous to a fluid dynamical
                    approach that incorporated real world behavior ofa driver but
                    also retained the key aspects of fluid dynamic studies.
          \end{description}
    \item Self-organization and phase transition in traffic-flow model of a
          two-lane roadway \cite{nagatani_self-organization_1993}
          \begin{description}
              \item Nagatani implements a two lane model as an extension of a one dimensional cellular automaton to monitor lane changing. Their model is essentially two 1-dimensional lattaces
              to represent a two lane roadway. Rules using arrows within the lattace were deteremined to allow cars to change lanes at certain times such as when a car infront was blocking
              the path to progress forward from either the right to left direcetion or vice versa. These rules were dependent on the time step. Overall the results showed that as the cars
              general velocity increased, there was a usage of lane shifting without cars obstructing progress. As density increases, the maximal velocity decreases. There was a "sweet-spot"
              indicating the optimal amount of lane changing within a set of density of traffic where lane changing peaked with a critical value of density. If the desnity fell or rose
              too high, this optimal amount of lane changing dropped. 
          \end{description}
    \item Multi-value cellular automata model for mixed bicycle flow
          \cite{jia_multi-value_2007}
          \begin{description}
              \item Instead of car and vehicle traffic like the other papers, this paper focuses on mixed bicycle flow using a multi valued cellular automata. Here the authors implemented two bicycles
              with different maximum speeds. The usage of bicycles to compare to vehicles have an emphasis on the behavioral and personality aspect that can be difficult to analyze with cars.
              The authors explain that young riders tend to ride at high speeds while older riders ride at a lower speed. These differences show that there is no maxium velocity. In this 
              model, bicycles can move to their next open site with faster bicycles moving with priority over slower bicycles. It was shown similarily to vehicles that slow bicycles that 
              congregate and occupy sites, block faster moving bicycles that cannot overtake and thus move in a platoon order. Increasing slow bicycle density will congest and bottleneck
              the simulation. One of the points that B. Jia and the authors mention that cannot be properly analysed is the nature of riders that group together because they are classmates
              or friends thus this simulation does have some flaws. Increasing randomization of the locations and density of slow riders tended to allow more free flow and less bottlenecking.
          \end{description}
    \item Cellular automaton model for bidirectional traffic
          \cite{simon_cellular_1998-1}
          \begin{description}
              \item Simon and Gutowitz have built upon the paper referenced above by
                    Nagel and Schreckenberg. Here they are implementing a
                    bidrectional traffic using a two lane road with traffic moving
                    in opposite direction in comparison to Nagel and Schreckenberg's
                    one lane cellular automaton. They have incorporated
                    inteteractions to simulate passing, as well as a distribution of
                    varying vehicle speeds. Once again, Simon and gutowitz, like
                    their reference model, had an emphasis on approximating the
                    behavior of real traffic and human behavior. They wanted to
                    address the issue of a one lane cellular atomaton where all
                    vehicles have a maximum velocity and thus the model
                    unrealistically follows a lead slow car, hence the need and
                    emphasis on passing. The bidirectional model varried in types
                    where there was varrying rules regarding passing. The
                    researchers have found that passing greatly increases fluidity
                    in traffic not seen in single dimensional cellular automaton and
                    greatly resembles real world traffic dynamics and flow.
          \end{description}
\end{itemize}

\section{Methods}
For this project, we plan to use Python to implement our cellular automata
traffic flow model. We may also include a GUI to visualize the traffic flow on
each time step. This section will be expanded as the project progresses.

\section{Milestones}
\begin{tabular}{|l|l|}
    \hline
    \textbf{Date} & \textbf{Milestone}                                      \\
    \hline
    $9/24/2021$   & \textbf{Project Deliverable 1}                          \\
    \hline
                  & Implement the original algorithm                        \\
    \hline
                  & Add weights to the cells                                \\
    \hline
                  & Add destination priorities                              \\
    \hline
                  & Model different traffic environments (one-way, etc.)    \\
    \hline
    $10/29/2021$  & \textbf{Project Deliverable 2}                          \\
    \hline
                  & Create different environments                           \\
    \hline
                  & Collect experimental results and present findings       \\
    \hline
                  & \textit{Optional} Add GUI to visualize the traffic flow \\
    \hline
\end{tabular}


\section{Project Management}
Our project has already set up weekly meetings in order to share progress and
updates, and we schedule ad-hoc meetings when necessary. Additionally, we have
our project inside of a github repository, where we can easily share our updated
code. We plan to make use of pair-programming sessions, where we can collaborate
while also sharing the load of programming.

\textbf{Members' Roles}
\begin{itemize}

    \item[-] \textbf{Andy Vu:} Programming, Writing
    \item[-] \textbf{Olivier Tran:} Programming, Writing, Submitting Documents
    \item[-] \textbf{Sanjeev Khemani:} Programming, Writing
    \item[-] \textbf{Noah Gardner:} Programming, Writing, Source Control
\end{itemize}

\bibliographystyle{ACM-Reference-Format}
\bibliography{report}

\end{document}
\endinput
